\section{Introduction}

In recent years, the intersection of robotics and agriculture has garnered significant attention due to its potential to revolutionize farming practices and address various challenges in crop cultivation and management. One of the critical issues faced by farmers worldwide is the proliferation of weeds in agricultural fields, which not only compete with crops for essential resources but also pose significant threats to livestock health and food safety. In particular, weeds such as Rumex have been identified as major nuisances in grasslands, where their presence can contaminate forage intended for grazing animals, including cows.

\vspace*{6mm}

The purity of grassland vegetation directly impacts the nutritional quality of forage consumed by livestock, thereby influencing the health and productivity of animals. The ingestion of contaminated forage, tainted with toxic or harmful weed species like Rumex, can lead to adverse health effects in cattle, affecting milk quality and quantity, as well as overall animal welfare. Therefore, the effective removal of weeds, particularly Rumex, from grasslands is imperative to ensure the integrity and safety of livestock feed, as well as the sustainability of agricultural operations.

\vspace*{6mm} 


In response to this pressing agricultural challenge posed by weed infestation in grasslands, this thesis endeavors to pioneer the development and implementation of an innovative coverage path planning (CPP) algorithm. Unlike conventional approaches, this algorithm prioritizes straight paths to optimize energy efficiency, thereby offering versatility across a spectrum of coverage path planning methodologies. While its primary application lies in weed removal within grasslands, its adaptability extends to diverse agricultural contexts, promising enhanced efficiency and sustainability in autonomous robotic operations. CPP plays a crucial role in guiding autonomous robotic systems to systematically traverse and cover an entire area of interest while minimizing overlap and maximizing efficiency. By leveraging advancements in robotics, artificial intelligence, and sensing technologies, CPP algorithms can enable autonomous agricultural robots to navigate complex terrain, identify weed-infested areas, and perform targeted weed removal tasks with precision and efficacy.

\vspace*{6mm} 

The primary objective of this research is to develop a coverage path planning (CPP) algorithm tailored for the intelligent and systematic removal of Rumex weeds from grasslands, thereby promoting the production of high-quality forage and safeguarding the health of grazing animals. This algorithm aims to optimize coverage efficiency, ensuring that all targeted weeds are effectively identified and removed by the robotic system. Through the integration of advanced robotics algorithms, the proposed approach seeks to enhance the efficacy of weed management practices while minimizing reliance on chemical herbicides and manual labor.


\vspace*{6mm} 

This thesis will explore various aspects of coverage path planning (CPP) in the context of agricultural robotics, including but not limited to:
\begin{itemize}

  \item Review of Literature: A comprehensive overview of path planning techniques, including coverage path planning (CPP) and alternatives accommodating non-holonomic constraints. It explores classical and heuristic algorithms, their optimization strategies, and computational complexities, underscoring the importance of effective path planning in agricultural robotics. Through synthesizing diverse research streams, this review informs the development of an innovative CPP algorithm for weed removal.
  
  \item Methodology: We provide an in-depth exploration of the proposed CPP algorithm designed specifically for autonomous weed removal operations. We elucidate the underlying design principles governing the algorithm's development, emphasizing its adaptability to varying terrain and distinct dataset. Furthermore, we outline the computational framework, detailing the algorithm's implementation and optimization strategies to ensure efficient coverage and weed removal. Additionally, we discuss sensor integration strategies, elucidating how sensor data is utilized for perception and decision-making during weed removal tasks. Finally, we delve into the decision-making processes guiding the robotic system's actions, encompassing strategies for global and local path planning and obstacle avoidance to maximize efficiency and effectiveness in weed management operations.
  
  \item Experimental Setup: This section outlines the key considerations for field deployment and validation of the CPP algorithm in real-world grassland environments. It discusses the selection of robotic hardware, focusing on factors such as mobility, durability, and compatibility with the intended application. Additionally, it elaborates on the sensor configurations essential for environmental perception and obstacle detection during weed removal operations. The section also addresses field testing protocols, detailing the procedures for data collection, performance evaluation, and validation of the CPP algorithm's efficacy under actual operating conditions.
  
  \item Results and Analysis: This section presents a comprehensive analysis of experimental findings obtained through both simulation and real-world field trials, encompassing various parameters crucial for field robotics. It includes quantitative assessments of weed removal efficiency, coverage completeness, Energy usage, and to name a few. Furthermore, it compares the performance of the proposed CPP algorithm with other state-of-the-art approaches, considering factors such as path optimality, computational efficiency, and adaptability to different environments. Through a rigorous evaluation framework, this analysis provides valuable insights into the effectiveness and applicability of the CPP algorithm in practical agricultural settings.
  
  \item Discussion and Implications: This section engages in a thorough analysis of the research findings in the context of existing literature, elucidating the alignment and disparities between the proposed CPP algorithm and previous studies. It identifies the strengths and limitations of the algorithm, considering factors such as computational efficiency, scalability, and adaptability to diverse agricultural environments. Furthermore, it explores potential applications of the CPP algorithm beyond weed removal, highlighting its broader implications for sustainable agriculture and livestock management practices. By synthesizing empirical evidence with theoretical insights, this discussion offers valuable perspectives on the future trajectory of autonomous robotics in agriculture.

\end{itemize}

\vspace*{6mm} 


By addressing these key components, this thesis aims to contribute to the advancement of autonomous weed management technologies in the agricultural sector, with a specific focus on improving the health and productivity of grassland ecosystems and ensuring the safety and quality of livestock feed. Through interdisciplinary collaboration and innovative engineering solutions, the integration of CPP algorithms into agricultural robotics holds promise for mitigating weed-related challenges and fostering sustainable farming practices for the benefit of farmers, consumers, and the environment alike.

\vspace*{6mm} 


By meticulously addressing these critical components, this thesis endeavors to significantly propel the evolution of autonomous weed management technologies within the agricultural domain. With a targeted emphasis on enhancing the health and productivity of grassland ecosystems, as well as safeguarding the safety and quality of livestock feed, the research aims to effect tangible improvements across multiple facets of agricultural sustainability.

\vspace*{6mm} 

Through a concerted effort to foster interdisciplinary collaboration and devise innovative engineering solutions, the integration of Coverage Path Planning (CPP) algorithms into agricultural robotics emerges as a beacon of hope for mitigating the myriad challenges posed by weed infestation. This holistic approach not only promises to alleviate immediate concerns for farmers but also extends its benefits to end consumers and the environment at large.

\vspace*{6mm} 

By empowering farmers with efficient and effective weed management tools, the research seeks to enhance agricultural productivity while reducing reliance on conventional, often environmentally detrimental, weed control methods. Moreover, by promoting sustainable farming practices, the integration of CPP algorithms holds the potential to foster long-term environmental stewardship and contribute to the preservation of natural ecosystems.

\vspace*{6mm} 

Ultimately, the overarching goal of this thesis is to usher in a new era of agricultural innovation—one characterized by the harmonious coexistence of technological advancement and environmental conservation. By harnessing the power of robotics and leveraging the principles of sustainable agriculture, the research endeavors to create a brighter future for farmers, consumers, and the planet alike.

\vspace*{6mm} 

need to include the algorithm of the paper as did in the first review paper 
