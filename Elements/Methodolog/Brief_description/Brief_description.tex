\section{Methodology}

\subsection{Brief Description}

Start with the algorithm, then you can modify it later.

\subsection{Data Preprocessing}


Data preprocessing is a pivotal step in the research methodology, especially in the context of coverage path planning for weed removal in agricultural fields. This section delineates the comprehensive preprocessing procedures employed to ensure the precision and efficacy of the data used in the subsequent analysis.

\vspace*{6mm}  


\textbf{Data Acquisition: }
The initial phase of preprocessing involves the acquisition of data pertaining to the weed positions within the field. This data collection is facilitated through the use of a drone equipped with a deep learning model capable of identifying and pinpointing the locations of weeds. The drone performs a systematic survey of the field, capturing high-resolution images and employing the deep learning model offline to detect weed positions. The resultant data is then utilized as input for the coverage path planning module.

\vspace*{6mm}  


Due to the developmental status of the deep learning model, the current implementation involves manual data acquisition using a real-time kinematic GPS (RTK) system. The RTK system offers centimeter-level accuracy in pinpointing weed locations, which, for the purpose of this research, is considered sufficiently precise. 

\vspace*{6mm}  


\textbf{Handling Data Uncertainty: }
In a practical scenario, the deep learning model's output would include positional data of weeds along with an associated uncertainty measure, reflecting the inherent imprecision of the system. However, given the reliance on RTK data for this research phase, we assume perfect positional accuracy. Future implementations incorporating the deep learning model will necessitate the adjustment of weed regions based on the uncertainty associated with each data point. This uncertainty measure will be used to define the regions of influence for each weed, ensuring that the coverage path planning algorithm accounts for the imprecision in the data. 


\vspace*{6mm}  


\textbf{Region Definition and Overlap Management: } Given the robot's extraction width of 60 cm, the field is divided into fixed regions. When data points from the drone indicate weed positions, these points often come with overlapping regions due to the growth patterns of weeds like Rumex, which tend to cluster.

\vspace*{6mm}  


Overlap Reduction: To address overlaps, the preprocessing algorithm calculates the centroids of overlapping regions. Points within overlapping regions are consolidated to avoid duplication, ensuring that each weed cluster is represented by a single centroid. This consolidation reduces redundancy and enhances the efficiency of the coverage path planning.

\vspace*{6mm}  


Non-Overlapping Weeds: For weeds whose regions do not overlap with others, the center of each weed is directly considered as a centroid. This step ensures that isolated weeds are accurately accounted for in the data set.

\vspace*{6mm}  


\textbf{Data Integration and Optimization: }
By processing the data to identify centroids and manage overlaps, we achieve a significant reduction in the total number of points. Preliminary results indicate a reduction of at least 40\% in the number of data points, optimizing the dataset for coverage path planning. This reduction not only minimizes the total traversal length required for weed removal but also conserves energy and reduces redundant revisits.

\vspace*{6mm}  


The processed data is then fed into the coverage path planning module, which utilizes the optimized set of points to devise an efficient path for weed removal, ensuring comprehensive coverage with minimal resource expenditure. The following figure illustrates the data preprocessing workflow, highlighting the steps taken to transform raw positional data into an optimized dataset ready for coverage path planning.

\vspace*{6mm}  



This detailed preprocessing approach ensures that the data fed into the coverage path planning algorithm is both accurate and efficient, laying a robust foundation for effective weed removal operations in agricultural fields.

\vspace*{6mm}  


I can also discuss two clustering techniques, first one implemented for amny clusters and the second one is the above one.  Also write in detail about the procedure followed for the above mentioned steps.
