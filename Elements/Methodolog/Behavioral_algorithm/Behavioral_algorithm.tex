
\subsection{Algorithm Description}

Behavioral Approach for Coverage Path Planning in Agricultural Fields

\vspace*{6mm}  

Following the preprocessing phase aimed at resolving the regions associated with designated points, the subsequent imperative lies in formulating an algorithm capable of comprehensively covering all identified points. With the completion of preprocessing, the focus narrows down to ensuring the precise coverage of all points to effectively identify and extract weed infestations. Although the agricultural robot in question operates under non-holonomic constraints, the overarching objective transcends mere efficiency and the identification of the shortest path. Instead, the primary emphasis lies in achieving exhaustive point coverage while strategically favoring approximately linear trajectories.

\vspace*{6mm}   

The rationale behind prioritizing linear trajectories over strictly adhering to non-holonomic paths is multifaceted. Firstly, the adoption of more curved paths significantly heightens the risk of grass damage, thereby undermining the fundamental objective of preserving grass quality. Given the paramount importance of maintaining optimal grass conditions within agricultural fields, any approach that compromises this aspect inherently fails to align with the core objectives. Secondly, the energy consumption associated with traversing curved paths is substantially higher compared to linear trajectories. For the specific agricultural robot under consideration, empirical estimates suggest that traversing an equivalent path length via curved trajectories incurs an energy expenditure four times greater than that of linear paths. Consequently, the central objective of the algorithm resides in identifying the shortest path capable of encompassing all designated points while mitigating curvature and prioritizing linear trajectories.

\vspace*{6mm}  

The development of this behavioral approach necessitates a nuanced understanding of agricultural terrain dynamics, robot kinematics, and energy efficiency considerations. By integrating these facets into the algorithmic design process, the resultant solution seeks to strike a delicate balance between point coverage efficacy, grass preservation, and energy optimization.

\vspace*{6mm}  


\textbf{Vision Cone Strategy: }

\vspace*{6mm}  


At this stage, having acquired comprehensive global information about the points in the field, the next step involves prioritizing straight paths while minimizing computational complexity and time. An innovative approach is employed wherein the robot is equipped with a vision cone mechanism. This vision cone is defined by two lines extending from the robot at a fixed angle and distance. The angle of these lines is determined based on the robot's minimum turning radius. For instance, for a robot with a minimum turning radius of 2 meters, the angle of the cone on either side is set at 11 degrees. The distance to the end of the cone depends on the operational area of the robot but is set to a fixed distance of 100 meters for this scenario.

\vspace*{6mm}  


The vision cone allows the robot to consider only those points within this cone from its current position as potential next travel points. This selective consideration significantly reduces the computational effort required to determine the path, as it disregards points outside the cone. By narrowing the focus to relevant points within the vision cone, computational efficiency is enhanced, thus making the vision cone an intelligent and effective strategy.

\vspace*{6mm}  


\textbf{Algorithmic Framework: }


\vspace*{6mm}  

The algorithmic framework is designed to facilitate comprehensive point coverage while minimizing curvature and prioritizing linear trajectories. The behavioral approach adopted for the algorithm is hierarchical, comprising three distinct behaviors that are sequentially activated. The transition from one behavior to the next is contingent upon the degree of point coverage achieved.

\begin{enumerate}
    \item \textbf{Initial Behavior: }The algorithm commences with the first behavior, designed to initiate coverage from the starting point. This stage focuses on covering a substantial portion of the field, leveraging the vision cone to select the next travel points and maintaining the priority on straight paths.
    
    \item \textbf{Intermediate Behavior:} Upon achieving a certain threshold of point coverage, the algorithm transitions to the second behavior. This intermediate stage aims to further optimize coverage by adjusting the strategy based on the points that remain. The robot continues to utilize the vision cone but adopt a slightly more flexible criteria for point selection to ensure efficient coverage progression.
    
    \item \textbf{Final Behavior:} Once the intermediate behavior reaches its saturation point—where additional coverage gains diminish—the algorithm shifts to the final behavior. This stage is designed to ensure complete, 100\% coverage of all remaining points. The final behavior will incorporate more refined strategies to target any residual areas, ensuring no point is left uncovered.
\end{enumerate}

\vspace*{6mm}   


This hierarchical behavioral algorithm ensures a methodical and efficient approach to coverage path planning. By starting with broad coverage strategies and progressively refining the approach, the algorithm effectively balances the need for comprehensive point coverage with the constraints of the robot's kinematic capabilities and the operational goal of preserving grass quality. The vision cone mechanism plays a pivotal role in this process, enhancing computational efficiency and enabling intelligent path selection. 

\vspace*{6mm}   


\textbf{Rationale for Hierarchical Approach: } 

\vspace*{6mm}   


The hierarchical approach is adopted to improve convergence rate and coverage efficiency. If a single algorithm or behavior were followed throughout, the robot would cover many points initially but gradually cover fewer points over time, reducing the algorithm's accuracy and increasing the overall path length and operational time. By transitioning between different behaviors, the algorithm can adapt to the changing density and distribution of points, maintaining high efficiency throughout the coverage process.

\vspace*{6mm}   


This hierarchical strategy ensures that the algorithm remains effective even as the number of uncovered points decreases. By tailoring the approach to the specific conditions encountered at each stage, the robot can optimize its path, reduce unnecessary movements, and maintain high precision in point coverage. This not only conserves energy but also preserves the quality of the grass by minimizing excessive traversal. The adaptive nature of the hierarchical approach thus represents a robust and efficient solution for coverage path planning in agricultural fields.

\vspace*{6mm}  

