\chapter{Experimental Setup}

\section{Environment}
The environment for our coverage motion planning algorithm is situated in a grass field characterized by small grass uniformly distributed across the area. The grass field is representative of typical agricultural settings, where weed management is essential for maintaining crop quality and productivity. As the grass matures, it is common for various unwanted plants, particularly weeds, to grow sporadically throughout the field. Our primary focus in this study is on the removal of Rumex (commonly known as dock weeds), which pose a significant challenge to farmers. However, the algorithm can be adapted to target other types of weeds based on specific requirements, the only that change would be the detection system to identify the target weed.

\section{Importance of Weed Removal}

We are concentrating on Rumex plants due to their detrimental impact on agricultural productivity and livestock health. Although Rumex plants are not inherently toxic, their presence in cattle feed can adversely affect the quality of milk production. Ensuring the purity of grass fed to cattle is crucial for producing high-quality, bio milk, which is not only more beneficial for human consumption but also promotes better health and well-being of the cattle. Pure grass feed leads to higher nutritional value in milk, contributing to improved dairy products. Therefore, effective weed management, particularly the removal of Rumex plants, is essential for maintaining the quality and productivity of grass fields.


\vspace{3mm} 


Traditionally, farmers have resorted to manually removing these weeds, a labor-intensive and time-consuming process. Manual removal becomes particularly arduous in large fields, imposing significant physical strain on farmers and limiting the efficiency of weed management. Automating this process with robotic systems offers a promising solution to enhance agricultural practices and improve farmers' quality of life.


\section{Constraints}

With a comprehensive understanding of the robot's capabilities and features, it is essential to delve into the constraints that arise due to its mechanical system and the nature of the field in which it operates. These constraints significantly influence the development of an effective motion planning algorithm.

\vspace{3mm}

\textbf{Non-Holonomic Nature:}
The robot is equipped with four regular wheels, limiting its movement capabilities. Unlike holonomic robots, which can move in any direction, our robot cannot move sideways. It is constrained to forward and backward movements and must make turns to change its direction. This non-holonomic nature adds a layer of complexity to the motion planning algorithm, as it must account for the robot's inability to change its heading instantaneously.

\vspace{3mm}

\textbf{Turning Radius:} The robot has a minimum turning radius of 2 meters, meaning it cannot make sharp turns. This constraint necessitates careful planning to ensure the robot can navigate around obstacles and reach all designated weed points without making turns that exceed its turning capabilities.

\vspace{3mm}

\textbf{Speed Variations:} The robot's speed must be carefully regulated to prevent damage to the grass and ensure precise weed removal. When moving in a straight path, the robot operates at a velocity of 0.8 m/s. However, when making a turn, the velocity is reduced to 0.4 m/s to maintain the 2-meter turning radius. This adjustment helps in maintaining stability and accuracy during turns, which is critical for avoiding collateral damage to the grass and ensuring efficient weed extraction.



\vspace{3mm}



With a clear understanding of the robot's constraints, we can now proceed to develop a motion planning algorithm that leverages these constraints to ensure comprehensive and efficient weed removal. By integrating considerations for non-holonomic movement, adaptive speed control, optimized coverage, obstacle avoidance, and environmental adaptability, the algorithm will facilitate the robot's autonomous operation, making the weed removal process more efficient and reducing the burden on farmers. This strategic approach not only enhances the robot's performance but also contributes to maintaining the health of the grass and improving the overall quality of the field.