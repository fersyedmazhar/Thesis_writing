%=== FRONT PART ===
%=== ABSTRCT ===

%\begin{center}
\chapter*{Abstract} 
% \section{Abstract}
\rhead{Abstract}
%\end{center}
% \addcontentsline{toc}{chapter}{Abstract}
% Insert a horizontal line
% \rule{\textwidth}{0.4pt}
% \par\noindent\rule{\textwidth}{0.4pt}

% \vspace*{5mm}
%===============================
% ABSTRACT GOES HERE - 250 words limit
%===============================
For centuries, agriculture has been fundamental to sustaining human life, particularly through the cultivation of grass to feed livestock. The quality of animal products is directly influenced by the quality of their feed, which is traditionally grown in agricultural fields. However, these fields are often infested with weeds, some of which are detrimental to livestock health and, consequently, the quality of animal products. Traditionally, farmers have used non-organic chemicals to control weeds, but this method is not conducive to producing organic feed. Manual weed removal is an alternative, though it is labor-intensive and inefficient.

To address this challenge, we propose an automated solution using robots equipped with an efficient path planning algorithm designed for comprehensive weed removal. This problem, known as coverage path planning (CPP), is critical for ensuring that all the weeds are covered efficiently. Our research introduces a novel behavioral algorithm tailored to perform complete CPP while adhering to the non-holonomic constraints of the robot. The algorithm incorporates three adaptive behaviors that evolve as coverage progresses, aiming to maximize coverage rate, reduce computation time, minimize field time, conserve energy, and shorten route length. Additionally, it prioritizes generating straight paths over curved ones to produce natural straight paths.

We developed two variants of the algorithm: one for fields without obstacles and another for fields with polygonal obstacles, such as houses and fences. Comparative analyses with the graph search and lawnmower approaches were conducted to evaluate the strengths and weaknesses of each method. Results demonstrate that our algorithm consistently maintains a high coverage rate, generates efficient paths, and minimizes energy consumption and route length. It also ensures hundred percent coverage of weeds. The algorithm's flexibility also allows customization for curved paths making it a versatile solution for other coverage path planning problems.

\vspace*{6mm}   

\hrule
%=============================== 

\vspace*{6mm}   

\textbf{Keywords:} Complete Coverage Path Planning, Obstacle Avoidance, Behavioral Algorithms, Non-Holonomic Constraints, Agricultural Robotics, Autonomous Navigation, Motion Planning, Continuous space, discrete space, occupancy grids, dubins path, Travelling Salesman Problem (TSP), Dubins Open Travelling Salesman Problem (DOTSP). 

%=== END OF ABSTRACT ===
